\documentclass[final,authoryear,3p,times]{elsarticle}

\usepackage{ecrc,natbib} 
\usepackage{mathtools}
\usepackage{listings}
\usepackage{multirow}
\usepackage{xcolor}

%% set the volume if you know. Otherwise `00'
\volume{00}

%% set the starting page if not 1
\firstpage{1}

%% Give the name of the journal
\journalname{Astronomy and Computing}

%% Give the author list to appear in the running head
%% Example \runauth{C.V. Radhakrishnan et al.}
\runauth{Sridhar et al.}

%% Give the abbreviation of the Journal.
\jid{procs}

%% Give a short journal name for the dummy logo (if needed)
\jnltitlelogo{Astronomy and computing}

\usepackage{amssymb}
\usepackage[figuresright]{rotating}

\newcommand\pasp{PASP }
\newcommand\pasa{PASA }
\newcommand\apj{Astrophysical Journal }
\newcommand\aaps{Astronomy and Astrophysics Supplement } 
\newcommand\aap{Astronomy and Astrophysics} 
\newcommand\aj{Astronomical Journal }
\newcommand\mnras{Monthly Notices of Royal Astronomical Society}
\definecolor{backcolour}{rgb}{0.95,0.95,0.92}

\begin{document}

\begin{frontmatter}

\title{RMSynth: A GPU-accelerated code for Rotation Measure Synthesis.}

\author[kap, ast]{Sarrvesh S. Sridhar}
\ead{sarrvesh.ss@gmail.com}
\author[csi, kap]{George Heald}

\address[kap]{Kapteyn Astronomical Institute, University of Groningen, Postbus 800, 9700AV Groningen, The Netherlands.}
\address[ast]{ASTRON, the Netherlands Institute for Radio Astronomy, Postbus 2, 7990 AA, Dwingeloo, The Netherlands.}
\address[csi]{csiro}

\begin{abstract}
Abstract
\end{abstract}

\begin{keyword}
one \sep  two  \sep three
\end{keyword}

\end{frontmatter}

%% main text
\section{Introduction}

\section{Overview of RM Synthesis}
\textcolor{red}{Brief mathematical overview of RM Synthesis... Look at Heald 2009 paper for inspiration.}
\textcolor{red}{Have a paragraph in the end about computational costs involved in RM Synthesis.}

\section{GPU implementation of RM Synthesis}

\section{Discussion}

\section{Conclusion}

\section*{Acknowledgements}

\bibliographystyle{model2-names}
\bibliography{rmsynthesis}

\end{document}
